%%===============================================================================================%%
%%=================================  博士论文主框架   ===========================================%%
%%=================================           ===========================================%%
%%=================================         ===========================================%%
%%=================================          ===========================================%%
%%=================================         ===========================================%%
%%=================================         ===========================================%%
%%=================================         ===========================================%%
%%=================================         ===========================================%%
%%===============================================================================================%%
\documentclass[twoside,doctor]{zjnuthesis}
% 博士只需将上述一行的 master 改为 doctor
% ==============================================================
% ==============================================================

% 自己需要增加什么 package 或修改什么设置的话,都放在这里吧。
\usepackage{enumerate}
% \usepackage{enumitem}
% \setlist[description,1]{itemindent=2pt,labelsep=2pt}
\usepackage{zhlipsum}
% \usepackage{etoc}
\usepackage{float}
\usepackage[caption=false,font=normalsize,labelfont=sf,textfont=sf]{subfig}
% \usepackage[noend]{algpseudocode}
\newcommand{\tabincell}[2]{\begin{tabular}{@{}#1@{}}#2\end{tabular}}
\allowdisplaybreaks
\raggedbottom
\setcounter{tocdepth}{2}
%%
%
%\setcounter{tocdepth}{3}:toc即table of content,表示目录显示的深度
%\setcounter{secnumdepth}{4}:secnum即section number,表示章节编号的深度
%其中的数字3和4表示深度,具体数字的含义如下:
%-1 part
%0 chapter
%1 section
%%2 subsection
%3 subsubsection
%4 paragraph
%5 subparagraph
%%
%%% ========================================================================== %%
%%% ============ 以下为封面内容,请根据实际情况填入 ========================== %%
%%% ========================================================================== %%
 % % 使用 tocloft 包来自定义目录格式
\usepackage{tocloft}

% % 设置不同级别标题的缩进
\setlength{\cftchapindent}{0em}        % 章节标题不缩进
\setlength{\cftsecindent}{2em}         % 一级节标题缩进 2em
\setlength{\cftsubsecindent}{4em}      % 二级节标题缩进 4em
\setlength{\cftsubsubsecindent}{6em}   % 三级节标题缩进 6em

% === 中文封面内容 ===%
% \SchoolCode{10345}         %学校代码
\ResearchType{应用研究}    %研究类型
\Cfdlevel{}       %密级, 博士填写
\FirstDiscip{计算机科学与技术}    %一级学科, 博士填写
\SecDiscip{计算机应用技术}    %二级学科, 博士填写
\ResearchDir{机器学习与模式识别}	  %研究方向, 博士填写
\college{计算机科学与技术学院}     %所在学院, 博士填写
\Title{
} %题目
\SubjectMajor{计算机科学与技术}                  %学科专业
\Grade{}               %年级
\StudentID{}     %学号
\Graduate{}             %作者
\Advisor{}     
% \Graduate{***}             %作者
% \Advisor{***}   
%指导教师
\Classification{TP391.4}      %中图分类号
\SubmitDate{2023~年~6~月~6~日}     %硕士论文提交时间
\submitDate{2025~年~6~月~10~日}     %博士论文提交时间

% === 中英文扉面内容 ===%
\Title{} %
\EnglishTitle{\textsc{generation}}
% \upperTitle{Research}
% \Author{***}                               %作者
% \EnglishAuthor{***}   
\Author{}                               %作者
\EnglishAuthor{n}                  %英文作者
% \Advisor{}                              %导师
\EnglishAdvisor{M}     %英文导师
% \Advisor{***}                              %导师
% \EnglishAdvisor{***}    
\Major{计算机科学与技术}                              %专业
\EnglishMajor{Computer Science and Technology}      %英文专业
\Degree{工学博士}                                 %学位
\EnglishDegree{Doctor of Engineering}                       %英文学位
\Institute{浙江师范大学}                      %授予单位
\EnglishInstitute{Zhejiang Normal University} %英文授予单位

\begin{document}
\newcommand{\ch}{\omega}
\newcommand{\nch}{\omega^*}
\newtheorem{observation}{观察}
\newtheorem{conjecture}{猜想}
\newtheorem*{claim}{断言}
\newtheorem{Remark}{注记}
\newtheorem{question}{问题}
% \newtheorem{technique}{技术}
\newtheorem{technique}[theorem]{\Heiti 技术}

% \newtheorem{algorithm}{算法}
\newtheorem{assumption}{\Heiti 假设}
% \newtheorem{proposition}{命题}
\newcommand{\ra}[1]{\renewcommand{\arraystretch}{#1}}
\counterwithin{algorithm}{chapter}

\floatname{algorithm}{算法}  % 将 "Algorithm" 改为 "算法"
\renewcommand{\thealgorithm}{\thechapter.\arabic{algorithm}}

\let\cleardoublepage\clearpage % 去除空白页
%=== 显示格式化中英文扉面 ===%
% \makeenglishtitle

% === 显示格式化封面 ===%
\maketitlecn

% === 显示格式化中英文扉面 ===%
\maketitleen
%%===========================================================================%%
%%========================= 目录部分 ========================================%%
%%===========================================================================%%

\frontmatter
% %把页码切换成罗马数字格式,并且不再对章进行自动编号
%   %=== 摘要 ===%
  %%---------------------------------------------------------------------------%%
%%---------------------------- 中英文摘要 -----------------------------------%%
%%---------------------------------------------------------------------------%%
\begin{Abstract}

该训练策略具有较强的泛用能力,生成质量高、模式覆盖全的样本。
%%%%%%% liyuan %%%%%%

\Keywords{图像生成模型;梯度流;生成对抗网络;扩散模型;梯度惩罚;训练策略}

\end{Abstract}


\begin{EnglishAbstract}

% To extend this analytical approach to general GAN, the study proposes a new nested annealed training strategy. Additionally, from the perspective of dynamic systems, we prove that this new training strategy is equivalent to the gradient field of functional gradient-flow-based GAN, sharing the same convergence properties. This new training strategy can be generalized to most GAN models, improving the generation quality of these models.


\EnglishKeywords{Image generation model; Gradient flow; Generative adversarial networks; Diffusion models; Gradient penalty; Training scheme}
\end{EnglishAbstract}


%   %=== 目录 ===%
  \tableofcontents

%   %=== 表格目录 ===%
  % \listoftables

%   %=== 插图目录 ===%
  % \listoffigures


% %%===========================================================================%%
% %%========================= 正文部分 ========================================%%
% %%===========================================================================%%

% %把页码切换成阿拉伯数字,并对章进行自动编号
\mainmatter

%   %=== 章节 ===%
  %%---------------------------------------------------------------------------%%
%%-------------------------- 第一章:绪论 -----------------------------------%%
%%---------------------------------------------------------------------------%%

\chapter{绪论}
\label{chapter:introduction}


\section{背景介绍及意义}

\subsection{选题理由}\label{liyou}

\begin{figure}[htbp]
    \centering
    \includegraphics[width=0.4\textwidth]{figures/wc/p-5.jpeg}
    \hspace{0.1in}
   \includegraphics[width=0.54\textwidth]{figures/wc/p-6.jpeg}
    \caption{文字引导的风格迁移任务,图像风格迁移任务。\label{transfer-1}}
\end{figure}

\subsection{选题意义}
图像生成模型作为目前计算机视觉的研究热点之一,它的理论研究和应用会为数学、计算机领域的研究和应用,甚至各行各业的发展带来重要的意义,为推动人类社会的进步奠定重要的基础。
% 本小结从科学意义和应用意义角度来阐述本研究选题的价值和意义。
% \subsubsection{科学意义}

% {\heiti(1)科学意义}

本研究选择基于梯度流的生成对抗网络的性能作为研究主题。该模型是生成对抗网络模型的一个分支,相比于一般的生成对抗网络模型,其特点是拥有较好的可解释性。但是,它的性能问题也十分明显,其生成的高分辨率图像质量较差、生成的图像对数据样本的类别覆盖性较差。并且,这类方法的泛用性也较弱,难以适用在不同类型的数据集、模型架构和训练方法之下。

为了解决以上问题,本研究需要基于数学分析、动力系统、机器学习、最优化理论、数理统计、运筹学等理论和技术,构建基于梯度流、对抗学习、扩散模型的图像生成模型,从隐式和显示概率分布变化的角度,利用动力系统、最优化理论解释和优化相应的数学模型,完成基于梯度流的图像生成模型的研究。通过新的研究思路和方法,丰富图像生成模型的理论和技术,进一步探究图像生成模型的内在规律,具有较好的科学研究意义。

% \subsubsection{应用意义}
% {\heiti(2)应用意义}

除了上述的科学意义,如选题意义~\ref{liyou} 中所述,图像生成模型在日常生活、工业应用、艺术创作等方面均有非常广阔的应用前景,例如
% 图\ref{application1} 和图
图
\ref{application2} 中所示。图像生成模型在风格迁移、多模态图像生成、图像编辑、手机拍照、文物复原、合成语音、视频、短视频风格变化、元宇宙3D物体重建等应用方面均有着非常重要的意义。

\section{研究现状与趋势}
\subsection{扩散模型的综述}
%

\section{研究目标与问题分析}


\subsection{本文研究目标}



\subsection{现有研究中存在的主要问题}


\section{研究内容与创新点}

现将本研究的内容与创新点总结为如下三个方面:
\subsection{研究内容}

\subsection{各研究内容之间的关系}

\section{本文组织结构}



  \chapter{基本理论与方法}
\label{chapter2}


\section{生成对抗网络的基本理论与方法}
\subsection{生成对抗网络的基础模型}\label{s2:gan_base}


\section{扩散模型的基本理论与方法}
\subsection{去噪扩散模型}\label{s2_ddpm}

  
% \chapter{基于新利普希茨约束的函数梯度流生成对抗网络模型}
\chapter{基于利普希茨约束的模式覆盖拓展理论模型}
\label{chapter3}



\section{序言}


\section{预备知识}



\section{小结}
本章中提供。

  % \chapter{扩散过程引导的基于梯度流的生成对抗网络的训练框架}
\chapter{基于扩散过程的图像质量提高训练框架}
\label{chapter4}
\section{序言}


\section{预备知识}
\subsection{生成对抗网络中的扩散过程}

\section{小结}
本章介绍了基
  % \chapter{函数梯度流引导的退火嵌套生成对抗网络训练策略}
\chapter{基于嵌套退火思想的泛用性增强训练策略}
\label{chapters/chapter5/}


\section{序言}
近年来,\cite{goodfellow2014generative}和扩散模型(DM)--已被广泛用于图像合成及相关任务。GAN 的工作原理是从低维向量空间采样。然后将这些向量输入神经网络,通过单一的前向传播生成图像。该框架以其合成高度逼真图像的能力和快速采样过程而著称。

\section{预备知识}

%
\begin{figure}[tp]
\centering 
\includegraphics[width=0.9\textwidth]{chapters/chapter5/anneal-cfg-idea.jpg}
\caption{
直观了解退火权重机制。
\label{Fig.Intuitive}}
\vspace{-0.2in}
\end{figure}

\subsection{退火采样算法}
郎之万动力学采样的方法可以参考章节~\ref{s2:sampling} 郎之万动力学采样中的内容。
而退火朗之万动力学采样(简称退火采样)是朗之万动力学采样的一种增强方法。

\section{
基于梯度流的生成对抗网络嵌套退火训练策略}
% \section{概率密度函数的分数和生成对抗网络模型判别器梯度的联系}
% 本节概述了所提出的方法,随后将描述其详细内容及与先前方法的关系。
\subsection{总览}\label{overview}
{\heiti(1)直观理解}

%
\begin{figure*}[!htb]
\centering 
\includegraphics[width=1\textwidth]{chapters/chapter5/nats_lsun_nd4_dg1.jpg}
\caption{本图展示了使用NATS训练的DDGAN在$256\times256$分辨率的LSUN Church数据集上的合成样本。
\label{Fig.nats_lsun256_nd4_1}}
\vspace{-0.2in}
\end{figure*}



\section{小结}
本章揭示了
  \chapter{总结与展望}
\label{chapter5}
本章首先对本文的主要研究成果进行了总结,对相关问题和方法做了分析和阐述,分析了工作中存在的不足,并对未来工作进行了展望。
\section{研究工作总结}
本文以

\section{研究中存在的不足与改进方向}
尽管本研究取得了一些进展,但是研究过程中依旧存着许多不足,现总结如下:



\section{未来研究展望}

从目前的互联网内容来看,图像生成模型已经对人们的日常生活产生了一定影响。在youtube、抖音以及国内的各类短视频平台上,已经出现了很多由图像、视频模型生成的娱乐视频。这些内容通过肉眼真假难辨,容易对人们产生不利的影响。未来人们应该更多关注生成模型的滥用问题,通过研究鉴定生成模型内容的判别模型,为未来的互联网安全提供保证。

% \section{研究心得与体会}




%   %=== 参考文献 ===%
%   % 使用 BibTeX
  % \include{references/references}
  \bibliography{paper}

% %%===========================================================================%%
% %%========================= 附件部分 ========================================%%
% %%===========================================================================%%

% %它不再对章进行自动编号,但不改变页码数字格式,不重设页码计数器
% \backmatter



%   %=== 致谢 ===%
  %%---------------------------------------------------------------------------%%
%%------------------------------- 致谢 --------------------------------------%%
%%---------------------------------------------------------------------------%%
\chapter*{致谢}
\addcontentsline{toc}{chapter}{致谢}
\pagestyle{plain}
\fancyfoot[C]{\thepage} % 页码
% \fancyhead[C]{}

% \begin{thanks}
\par 回眸五年光阴,这是我
人生中最值得铭记的时光。


% \end{thanks}

%   %=== 发表文章目录 ===%
  %%---------------------------------------------------------------------------%%
%%----------------------------- 研究成果 ------------------------------------%%
%%---------------------------------------------------------------------------%%
\chapter*{攻读学位期间取得的研究成果}
\addcontentsline{toc}{chapter}{攻读学位期间取得的研究成果}

% \chapter*{攻读博士学位期间发表的学术论文}
% \addcontentsline{toc}{chapter}{攻读博士学位期间发表的学术论文}

\begin{enumerate}[{[}1{]}]

\item 

\item 
    
\item 
    
\item 
    
\end{enumerate}
% \begin{publications}{20}

% \item ******





% \end{publications}


%   %=== 学位论文独创性声明和学位论文使用授权声明 ===%
  %%% ========================================================================== %%
%%% ============== 浙江师范大学学位论文诚信承诺书 ============================ %%
%%% ========================================================================== %%

%% 诚信承诺书
\chapter*{浙江师范大学学位论文诚信承诺书}
\addcontentsline{toc}{chapter}{浙江师范大学学位论文诚信承诺书}

{\xiaosan\linespread{1.5}\selectfont
  我承诺自觉遵守《浙江师范大学研究生学术道德规范管理条例》。我的学位论文中凡引用他人已经发表或未发表的成果、数据、观点等,均已明确注明并详细列出有关文献的名称、作者、年份、刊物名称和出版文献的出版机构、出版地和版次等内容。论文中未注明的内容为本人的研究成果。

  如有违反,本人接受处罚并承担一切责任。      

  \vspace{20pt}
  \begin{flushright}
    承诺人(研究生): \phantom{\hspace{100pt}} \\[20pt]
    指导教师: \phantom{\hspace{100pt}}
  \end{flushright}
}


%%% ========================================================================== %%
%%% ========= 学位论文独创性声明和学位论文使用授权声明 ======================= %%
%%% ========================================================================== %%

%% 以下这四行代码是在该页面之前加入一页空白页!!
% \newpage
% \thispagestyle{empty}
% \
% \newpage

\begin{flushleft}
  \begin{minipage}{15cm}
    \begin{OriginalityStatements}
      \setlength{\parindent}{2em}\xiaosi
      本人声明所呈交的学位论文是我个人在导师指导下进行的研究工作及取得的研究成果。论文中除了特别加以标注和致谢的地方外,不包含其他人或其他机构已经发表或撰写过的研究成果。其他同志对本研究的启发和所做的贡献均已在论文中作了明确的声明并表示了谢意。本人完全意识到本声明的法律结果由本人承担。
      \newline
      \newline

      \noindent
      作者签名:\hspace*{60pt}
      \hfill
      日期:\hspace*{40pt}2025
      年  \hspace*{20pt}6
      月  \hspace*{20pt}10 日
    \end{OriginalityStatements}

    \vspace{20ex}
    
    \begin{LicenseStatements}
      \setlength{\parindent}{2em}\xiaosi
      本人完全了解浙江师范大学有关保留、使用学位论文的规定,即:学校有权保留并向国家有关机关或机构送交论文的复印件和电子文档,允许论文被查阅和借阅,可以采用影印、缩印或扫描等手段保存、汇编学位论文。同意浙江师范大学可以用不同方式在不同媒体上发表、传播论文的全部或部分内容。
      
      保密的学位论文在解密后遵守此协议。
      \newline
      \newline

      \noindent
      作者签名:\hspace*{60pt}
      导师签名:\hfill
      日期:\hspace*{40pt}2025
      年 \hspace*{20pt}6
      月 \hspace*{20pt}10 日

    \end{LicenseStatements}
  \end{minipage}
\end{flushleft}

%   %=== 学位论文诚信承诺书 ===%
  \include{chapters/chengnuoshu}
% 不要修改以下内容
%%% ========================================================================== %%
%%% ============== 浙江师范大学学位论文诚信承诺书 ============================ %%
%%% ========================================================================== %%

%% 诚信承诺书
\chapter*{浙江师范大学学位论文诚信承诺书}
\addcontentsline{toc}{chapter}{浙江师范大学学位论文诚信承诺书}

{\xiaosan\linespread{1.5}\selectfont
  我承诺自觉遵守《浙江师范大学研究生学术道德规范管理条例》。我的学位论文中凡引用他人已经发表或未发表的成果、数据、观点等,均已明确注明并详细列出有关文献的名称、作者、年份、刊物名称和出版文献的出版机构、出版地和版次等内容。论文中未注明的内容为本人的研究成果。

  如有违反,本人接受处罚并承担一切责任。      

  \vspace{20pt}
  \begin{flushright}
    承诺人(研究生): \phantom{\hspace{100pt}} \\[20pt]
    指导教师: \phantom{\hspace{100pt}}
  \end{flushright}
}


%%% ========================================================================== %%
%%% ========= 学位论文独创性声明和学位论文使用授权声明 ======================= %%
%%% ========================================================================== %%

%% 以下这四行代码是在该页面之前加入一页空白页!!
% \newpage
% \thispagestyle{empty}
% \
% \newpage

\begin{flushleft}
  \begin{minipage}{15cm}
    \begin{OriginalityStatements}
      \setlength{\parindent}{2em}\xiaosi
      本人声明所呈交的学位论文是我个人在导师指导下进行的研究工作及取得的研究成果。论文中除了特别加以标注和致谢的地方外,不包含其他人或其他机构已经发表或撰写过的研究成果。其他同志对本研究的启发和所做的贡献均已在论文中作了明确的声明并表示了谢意。本人完全意识到本声明的法律结果由本人承担。
      \newline
      \newline

      \noindent
      作者签名:\hspace*{60pt}
      \hfill
      日期:\hspace*{40pt}2025
      年  \hspace*{20pt}6
      月  \hspace*{20pt}10 日
    \end{OriginalityStatements}

    \vspace{20ex}
    
    \begin{LicenseStatements}
      \setlength{\parindent}{2em}\xiaosi
      本人完全了解浙江师范大学有关保留、使用学位论文的规定,即:学校有权保留并向国家有关机关或机构送交论文的复印件和电子文档,允许论文被查阅和借阅,可以采用影印、缩印或扫描等手段保存、汇编学位论文。同意浙江师范大学可以用不同方式在不同媒体上发表、传播论文的全部或部分内容。
      
      保密的学位论文在解密后遵守此协议。
      \newline
      \newline

      \noindent
      作者签名:\hspace*{60pt}
      导师签名:\hfill
      日期:\hspace*{40pt}2025
      年 \hspace*{20pt}6
      月 \hspace*{20pt}10 日

    \end{LicenseStatements}
  \end{minipage}
\end{flushleft}


\end{document}
