%%---------------------------------------------------------------------------%%
%%-------------------------- 第一章:绪论 -----------------------------------%%
%%---------------------------------------------------------------------------%%

\chapter{绪论}
\label{chapter:introduction}


\section{背景介绍及意义}

\subsection{选题理由}\label{liyou}

\begin{figure}[htbp]
    \centering
    \includegraphics[width=0.4\textwidth]{figures/wc/p-5.jpeg}
    \hspace{0.1in}
   \includegraphics[width=0.54\textwidth]{figures/wc/p-6.jpeg}
    \caption{文字引导的风格迁移任务,图像风格迁移任务。\label{transfer-1}}
\end{figure}

\subsection{选题意义}
图像生成模型作为目前计算机视觉的研究热点之一,它的理论研究和应用会为数学、计算机领域的研究和应用,甚至各行各业的发展带来重要的意义,为推动人类社会的进步奠定重要的基础。
% 本小结从科学意义和应用意义角度来阐述本研究选题的价值和意义。
% \subsubsection{科学意义}

% {\heiti(1)科学意义}

本研究选择基于梯度流的生成对抗网络的性能作为研究主题。该模型是生成对抗网络模型的一个分支,相比于一般的生成对抗网络模型,其特点是拥有较好的可解释性。但是,它的性能问题也十分明显,其生成的高分辨率图像质量较差、生成的图像对数据样本的类别覆盖性较差。并且,这类方法的泛用性也较弱,难以适用在不同类型的数据集、模型架构和训练方法之下。

为了解决以上问题,本研究需要基于数学分析、动力系统、机器学习、最优化理论、数理统计、运筹学等理论和技术,构建基于梯度流、对抗学习、扩散模型的图像生成模型,从隐式和显示概率分布变化的角度,利用动力系统、最优化理论解释和优化相应的数学模型,完成基于梯度流的图像生成模型的研究。通过新的研究思路和方法,丰富图像生成模型的理论和技术,进一步探究图像生成模型的内在规律,具有较好的科学研究意义。

% \subsubsection{应用意义}
% {\heiti(2)应用意义}

除了上述的科学意义,如选题意义~\ref{liyou} 中所述,图像生成模型在日常生活、工业应用、艺术创作等方面均有非常广阔的应用前景,例如
% 图\ref{application1} 和图
图
\ref{application2} 中所示。图像生成模型在风格迁移、多模态图像生成、图像编辑、手机拍照、文物复原、合成语音、视频、短视频风格变化、元宇宙3D物体重建等应用方面均有着非常重要的意义。

\section{研究现状与趋势}
\subsection{扩散模型的综述}
%

\section{研究目标与问题分析}


\subsection{本文研究目标}



\subsection{现有研究中存在的主要问题}


\section{研究内容与创新点}

现将本研究的内容与创新点总结为如下三个方面:
\subsection{研究内容}

\subsection{各研究内容之间的关系}

\section{本文组织结构}


