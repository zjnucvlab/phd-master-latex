% \chapter{函数梯度流引导的退火嵌套生成对抗网络训练策略}
\chapter{基于嵌套退火思想的泛用性增强训练策略}
\label{chapters/chapter5/}


\section{序言}
近年来,\cite{goodfellow2014generative}和扩散模型(DM)--已被广泛用于图像合成及相关任务。GAN 的工作原理是从低维向量空间采样。然后将这些向量输入神经网络,通过单一的前向传播生成图像。该框架以其合成高度逼真图像的能力和快速采样过程而著称。

\section{预备知识}

%
\begin{figure}[tp]
\centering 
\includegraphics[width=0.9\textwidth]{chapters/chapter5/anneal-cfg-idea.jpg}
\caption{
直观了解退火权重机制。
\label{Fig.Intuitive}}
\vspace{-0.2in}
\end{figure}

\subsection{退火采样算法}
郎之万动力学采样的方法可以参考章节~\ref{s2:sampling} 郎之万动力学采样中的内容。
而退火朗之万动力学采样(简称退火采样)是朗之万动力学采样的一种增强方法。

\section{
基于梯度流的生成对抗网络嵌套退火训练策略}
% \section{概率密度函数的分数和生成对抗网络模型判别器梯度的联系}
% 本节概述了所提出的方法,随后将描述其详细内容及与先前方法的关系。
\subsection{总览}\label{overview}
{\heiti(1)直观理解}

%
\begin{figure*}[!htb]
\centering 
\includegraphics[width=1\textwidth]{chapters/chapter5/nats_lsun_nd4_dg1.jpg}
\caption{本图展示了使用NATS训练的DDGAN在$256\times256$分辨率的LSUN Church数据集上的合成样本。
\label{Fig.nats_lsun256_nd4_1}}
\vspace{-0.2in}
\end{figure*}



\section{小结}
本章揭示了